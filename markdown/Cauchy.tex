% Options for packages loaded elsewhere
\PassOptionsToPackage{unicode}{hyperref}
\PassOptionsToPackage{hyphens}{url}
\documentclass[
]{article}
\usepackage{xcolor}
\usepackage{amsmath,amssymb}
\setcounter{secnumdepth}{-\maxdimen} % remove section numbering
\usepackage{iftex}
\ifPDFTeX
  \usepackage[T1]{fontenc}
  \usepackage[utf8]{inputenc}
  \usepackage{textcomp} % provide euro and other symbols
\else % if luatex or xetex
  \usepackage{unicode-math} % this also loads fontspec
  \defaultfontfeatures{Scale=MatchLowercase}
  \defaultfontfeatures[\rmfamily]{Ligatures=TeX,Scale=1}
\fi
\usepackage{lmodern}
\ifPDFTeX\else
  % xetex/luatex font selection
\fi
% Use upquote if available, for straight quotes in verbatim environments
\IfFileExists{upquote.sty}{\usepackage{upquote}}{}
\IfFileExists{microtype.sty}{% use microtype if available
  \usepackage[]{microtype}
  \UseMicrotypeSet[protrusion]{basicmath} % disable protrusion for tt fonts
}{}
\makeatletter
\@ifundefined{KOMAClassName}{% if non-KOMA class
  \IfFileExists{parskip.sty}{%
    \usepackage{parskip}
  }{% else
    \setlength{\parindent}{0pt}
    \setlength{\parskip}{6pt plus 2pt minus 1pt}}
}{% if KOMA class
  \KOMAoptions{parskip=half}}
\makeatother
\usepackage{color}
\usepackage{fancyvrb}
\newcommand{\VerbBar}{|}
\newcommand{\VERB}{\Verb[commandchars=\\\{\}]}
\DefineVerbatimEnvironment{Highlighting}{Verbatim}{commandchars=\\\{\}}
% Add ',fontsize=\small' for more characters per line
\newenvironment{Shaded}{}{}
\newcommand{\AlertTok}[1]{\textcolor[rgb]{1.00,0.00,0.00}{\textbf{#1}}}
\newcommand{\AnnotationTok}[1]{\textcolor[rgb]{0.38,0.63,0.69}{\textbf{\textit{#1}}}}
\newcommand{\AttributeTok}[1]{\textcolor[rgb]{0.49,0.56,0.16}{#1}}
\newcommand{\BaseNTok}[1]{\textcolor[rgb]{0.25,0.63,0.44}{#1}}
\newcommand{\BuiltInTok}[1]{\textcolor[rgb]{0.00,0.50,0.00}{#1}}
\newcommand{\CharTok}[1]{\textcolor[rgb]{0.25,0.44,0.63}{#1}}
\newcommand{\CommentTok}[1]{\textcolor[rgb]{0.38,0.63,0.69}{\textit{#1}}}
\newcommand{\CommentVarTok}[1]{\textcolor[rgb]{0.38,0.63,0.69}{\textbf{\textit{#1}}}}
\newcommand{\ConstantTok}[1]{\textcolor[rgb]{0.53,0.00,0.00}{#1}}
\newcommand{\ControlFlowTok}[1]{\textcolor[rgb]{0.00,0.44,0.13}{\textbf{#1}}}
\newcommand{\DataTypeTok}[1]{\textcolor[rgb]{0.56,0.13,0.00}{#1}}
\newcommand{\DecValTok}[1]{\textcolor[rgb]{0.25,0.63,0.44}{#1}}
\newcommand{\DocumentationTok}[1]{\textcolor[rgb]{0.73,0.13,0.13}{\textit{#1}}}
\newcommand{\ErrorTok}[1]{\textcolor[rgb]{1.00,0.00,0.00}{\textbf{#1}}}
\newcommand{\ExtensionTok}[1]{#1}
\newcommand{\FloatTok}[1]{\textcolor[rgb]{0.25,0.63,0.44}{#1}}
\newcommand{\FunctionTok}[1]{\textcolor[rgb]{0.02,0.16,0.49}{#1}}
\newcommand{\ImportTok}[1]{\textcolor[rgb]{0.00,0.50,0.00}{\textbf{#1}}}
\newcommand{\InformationTok}[1]{\textcolor[rgb]{0.38,0.63,0.69}{\textbf{\textit{#1}}}}
\newcommand{\KeywordTok}[1]{\textcolor[rgb]{0.00,0.44,0.13}{\textbf{#1}}}
\newcommand{\NormalTok}[1]{#1}
\newcommand{\OperatorTok}[1]{\textcolor[rgb]{0.40,0.40,0.40}{#1}}
\newcommand{\OtherTok}[1]{\textcolor[rgb]{0.00,0.44,0.13}{#1}}
\newcommand{\PreprocessorTok}[1]{\textcolor[rgb]{0.74,0.48,0.00}{#1}}
\newcommand{\RegionMarkerTok}[1]{#1}
\newcommand{\SpecialCharTok}[1]{\textcolor[rgb]{0.25,0.44,0.63}{#1}}
\newcommand{\SpecialStringTok}[1]{\textcolor[rgb]{0.73,0.40,0.53}{#1}}
\newcommand{\StringTok}[1]{\textcolor[rgb]{0.25,0.44,0.63}{#1}}
\newcommand{\VariableTok}[1]{\textcolor[rgb]{0.10,0.09,0.49}{#1}}
\newcommand{\VerbatimStringTok}[1]{\textcolor[rgb]{0.25,0.44,0.63}{#1}}
\newcommand{\WarningTok}[1]{\textcolor[rgb]{0.38,0.63,0.69}{\textbf{\textit{#1}}}}
\setlength{\emergencystretch}{3em} % prevent overfull lines
\providecommand{\tightlist}{%
  \setlength{\itemsep}{0pt}\setlength{\parskip}{0pt}}
\usepackage{bookmark}
\IfFileExists{xurl.sty}{\usepackage{xurl}}{} % add URL line breaks if available
\urlstyle{same}
\hypersetup{
  hidelinks,
  pdfcreator={LaTeX via pandoc}}

\author{}
\date{}

\begin{document}

The difference of exactly a factor of 1/2 arises because
Julia\textquotesingle s \texttt{SingularIntegralEquations.jl} and the
standard Mathematica \texttt{Integrate} function are using
\textbf{different, but equally valid, definitions} of the Cauchy
Transform.

Your Mathematica code implements what I\textquotesingle ll call the
\textbf{"Analyst\textquotesingle s Definition"}, which is the one you
linked to earlier for the Kramers-Kronig relations:

\[C_{\text{Ana}}[f](z) = \frac{1}{\pi i} \int \frac{f(t)}{t-z} \, dt\]

Julia\textquotesingle s \texttt{SingularIntegralEquations.cauchy}
function uses the \textbf{"Integral Equation Definition"} common in
singular integral equation theory:

\[C_{\text{IE}}[f](z) = \frac{1}{2\pi i} \int \frac{f(t)}{t-z} \, dt\]

Therefore, the relation is simply:

\[C_{\text{Ana}}[f](z) = 2 \times C_{\text{IE}}[f](z)\]

Your results confirm this: \texttt{0.479351...} (Julia) is half of what
Mathematica gives for the same calculation.

\subsubsection{🧮 Why the Factor of 2 Matters: Connection to the
Plemelj-Sokhotsky
Formula}\label{ux1f9ee-why-the-factor-of-2-matters-connection-to-the-plemelj-sokhotsky-formula}

The "Integral Equation Definition" (\(1/(2\pi i)\)) is often preferred
when connecting to the \textbf{Plemelj-Sokhotsky formula}, which
describes the limiting values of a Cauchy integral as you approach the
real axis. It states:

\[\lim_{\epsilon \to 0^+} \frac{1}{2\pi i} \int \frac{f(t)}{t - (x \pm i\epsilon)} \, dt = \pm \frac{1}{2} f(x) + \frac{1}{2\pi i} \text{P.V.} \int \frac{f(t)}{t - x} \, dt\]

Here, the jump (the \texttt{±\ 1/2\ f(x)}) is nice and clean. The
"Analyst\textquotesingle s Definition" would have a jump of
\texttt{±\ f(x)}.

\subsubsection{🔄 Adapting Your Codes for
Consistency}\label{ux1f504-adapting-your-codes-for-consistency}

You can easily make either code match the other\textquotesingle s
convention.

\textbf{1. To make Mathematica match the Julia result:}\\
Simply multiply your Mathematica integral by \texttt{1/2}.

\begin{Shaded}
\begin{Highlighting}[]
\CommentTok{(* Mathematica code matching Julia\textquotesingle{}s cauchy(f, z) *)}
\NormalTok{CauchyJulia}\OperatorTok{[}\AttributeTok{z\_}\OperatorTok{]} \ExtensionTok{:=} \FunctionTok{Integrate}\OperatorTok{[}\FunctionTok{Exp}\OperatorTok{[}\FunctionTok{t}\OperatorTok{]}\SpecialCharTok{/}\NormalTok{(}\FunctionTok{t} \SpecialCharTok{{-}} \FunctionTok{z}\NormalTok{)}\OperatorTok{,} \OperatorTok{\{}\FunctionTok{t}\OperatorTok{,} \SpecialCharTok{{-}}\DecValTok{1}\OperatorTok{,} \DecValTok{1}\OperatorTok{\}]} \SpecialCharTok{/}\NormalTok{ (}\DecValTok{2} \SpecialCharTok{*} \FunctionTok{Pi} \SpecialCharTok{*} \FunctionTok{I}\NormalTok{)}
\end{Highlighting}
\end{Shaded}

\textbf{2. To make Julia match your Mathematica result:}\\
Multiply Julia\textquotesingle s result by \texttt{2}.

\begin{Shaded}
\begin{Highlighting}[]
\CommentTok{\# Julia code matching your Mathematica definition}
\NormalTok{C\_val\_analyst }\OperatorTok{=} \FloatTok{2} \OperatorTok{*} \FunctionTok{cauchy}\NormalTok{(f, z)}
\CommentTok{\# This should now give: \textasciitilde{}0.958702... + 0.0539166...im}
\end{Highlighting}
\end{Shaded}

\subsubsection{💡 Which One Should You
Use?}\label{ux1f4a1-which-one-should-you-use}

It depends on your field:

\begin{itemize}
\item
  \textbf{Complex Analysis / Kramers-Kronig Relations}: Stick with the
  \textbf{Analyst\textquotesingle s Definition} (\(1/(\pi i)\)). This is
  standard in physics and optics.
\item
  \textbf{Singular Integral Equations / Fracture Mechanics}: The
  \textbf{Integral Equation Definition} (\(1/(2\pi i)\)) is more common.
  This is the world of the Kolosov-Muskhelishvili potentials we
  discussed earlier.
\end{itemize}

The key is to be aware of the definition in the papers or textbooks you
are following.

\subsubsection{✅ Summary and
Verification}\label{ux2705-summary-and-verification}

You can quickly verify the relationship by checking the limiting value
as \texttt{z} approaches the real axis. Using the Plemelj-Sokhotsky
formula with Julia\textquotesingle s definition:

\begin{itemize}
\item
  For \texttt{z\ -\textgreater{}\ x\ +\ i*0+} (from above), the real
  part should approach \texttt{+1/2\ *\ f(x)}.
\item
  For \texttt{z\ -\textgreater{}\ x\ -\ i*0+} (from below), the real
  part should approach \texttt{-1/2\ *\ f(x)}.
\end{itemize}

In summary, the \textbf{difference is not an error but a convention}.
Once you account for the factor of 2, both codes are correct. For your
engineering work in fracture mechanics (which heavily uses singular
integrals), Julia\textquotesingle s built-in convention is likely the
more direct one to use.

If you are comparing results from different sources, always check which
definition they use.

\begin{Shaded}
\begin{Highlighting}[]

\end{Highlighting}
\end{Shaded}

%% Options for packages loaded elsewhere
\PassOptionsToPackage{unicode}{hyperref}
\PassOptionsToPackage{hyphens}{url}
\documentclass[
]{article}
\usepackage{xcolor}
\usepackage{amsmath,amssymb}
\setcounter{secnumdepth}{-\maxdimen} % remove section numbering
\usepackage{iftex}
\ifPDFTeX
  \usepackage[T1]{fontenc}
  \usepackage[utf8]{inputenc}
  \usepackage{textcomp} % provide euro and other symbols
\else % if luatex or xetex
  \usepackage{unicode-math} % this also loads fontspec
  \defaultfontfeatures{Scale=MatchLowercase}
  \defaultfontfeatures[\rmfamily]{Ligatures=TeX,Scale=1}
\fi
\usepackage{lmodern}
\ifPDFTeX\else
  % xetex/luatex font selection
\fi
% Use upquote if available, for straight quotes in verbatim environments
\IfFileExists{upquote.sty}{\usepackage{upquote}}{}
\IfFileExists{microtype.sty}{% use microtype if available
  \usepackage[]{microtype}
  \UseMicrotypeSet[protrusion]{basicmath} % disable protrusion for tt fonts
}{}
\makeatletter
\@ifundefined{KOMAClassName}{% if non-KOMA class
  \IfFileExists{parskip.sty}{%
    \usepackage{parskip}
  }{% else
    \setlength{\parindent}{0pt}
    \setlength{\parskip}{6pt plus 2pt minus 1pt}}
}{% if KOMA class
  \KOMAoptions{parskip=half}}
\makeatother
\usepackage{color}
\usepackage{fancyvrb}
\newcommand{\VerbBar}{|}
\newcommand{\VERB}{\Verb[commandchars=\\\{\}]}
\DefineVerbatimEnvironment{Highlighting}{Verbatim}{commandchars=\\\{\}}
% Add ',fontsize=\small' for more characters per line
\newenvironment{Shaded}{}{}
\newcommand{\AlertTok}[1]{\textcolor[rgb]{1.00,0.00,0.00}{\textbf{#1}}}
\newcommand{\AnnotationTok}[1]{\textcolor[rgb]{0.38,0.63,0.69}{\textbf{\textit{#1}}}}
\newcommand{\AttributeTok}[1]{\textcolor[rgb]{0.49,0.56,0.16}{#1}}
\newcommand{\BaseNTok}[1]{\textcolor[rgb]{0.25,0.63,0.44}{#1}}
\newcommand{\BuiltInTok}[1]{\textcolor[rgb]{0.00,0.50,0.00}{#1}}
\newcommand{\CharTok}[1]{\textcolor[rgb]{0.25,0.44,0.63}{#1}}
\newcommand{\CommentTok}[1]{\textcolor[rgb]{0.38,0.63,0.69}{\textit{#1}}}
\newcommand{\CommentVarTok}[1]{\textcolor[rgb]{0.38,0.63,0.69}{\textbf{\textit{#1}}}}
\newcommand{\ConstantTok}[1]{\textcolor[rgb]{0.53,0.00,0.00}{#1}}
\newcommand{\ControlFlowTok}[1]{\textcolor[rgb]{0.00,0.44,0.13}{\textbf{#1}}}
\newcommand{\DataTypeTok}[1]{\textcolor[rgb]{0.56,0.13,0.00}{#1}}
\newcommand{\DecValTok}[1]{\textcolor[rgb]{0.25,0.63,0.44}{#1}}
\newcommand{\DocumentationTok}[1]{\textcolor[rgb]{0.73,0.13,0.13}{\textit{#1}}}
\newcommand{\ErrorTok}[1]{\textcolor[rgb]{1.00,0.00,0.00}{\textbf{#1}}}
\newcommand{\ExtensionTok}[1]{#1}
\newcommand{\FloatTok}[1]{\textcolor[rgb]{0.25,0.63,0.44}{#1}}
\newcommand{\FunctionTok}[1]{\textcolor[rgb]{0.02,0.16,0.49}{#1}}
\newcommand{\ImportTok}[1]{\textcolor[rgb]{0.00,0.50,0.00}{\textbf{#1}}}
\newcommand{\InformationTok}[1]{\textcolor[rgb]{0.38,0.63,0.69}{\textbf{\textit{#1}}}}
\newcommand{\KeywordTok}[1]{\textcolor[rgb]{0.00,0.44,0.13}{\textbf{#1}}}
\newcommand{\NormalTok}[1]{#1}
\newcommand{\OperatorTok}[1]{\textcolor[rgb]{0.40,0.40,0.40}{#1}}
\newcommand{\OtherTok}[1]{\textcolor[rgb]{0.00,0.44,0.13}{#1}}
\newcommand{\PreprocessorTok}[1]{\textcolor[rgb]{0.74,0.48,0.00}{#1}}
\newcommand{\RegionMarkerTok}[1]{#1}
\newcommand{\SpecialCharTok}[1]{\textcolor[rgb]{0.25,0.44,0.63}{#1}}
\newcommand{\SpecialStringTok}[1]{\textcolor[rgb]{0.73,0.40,0.53}{#1}}
\newcommand{\StringTok}[1]{\textcolor[rgb]{0.25,0.44,0.63}{#1}}
\newcommand{\VariableTok}[1]{\textcolor[rgb]{0.10,0.09,0.49}{#1}}
\newcommand{\VerbatimStringTok}[1]{\textcolor[rgb]{0.25,0.44,0.63}{#1}}
\newcommand{\WarningTok}[1]{\textcolor[rgb]{0.38,0.63,0.69}{\textbf{\textit{#1}}}}
\setlength{\emergencystretch}{3em} % prevent overfull lines
\providecommand{\tightlist}{%
  \setlength{\itemsep}{0pt}\setlength{\parskip}{0pt}}
\usepackage{bookmark}
\IfFileExists{xurl.sty}{\usepackage{xurl}}{} % add URL line breaks if available
\urlstyle{same}
\hypersetup{
  hidelinks,
  pdfcreator={LaTeX via pandoc}}

\author{}
\date{}

\begin{document}

The difference of exactly a factor of 1/2 arises because
Julia\textquotesingle s \texttt{SingularIntegralEquations.jl} and the
standard Mathematica \texttt{Integrate} function are using
\textbf{different, but equally valid, definitions} of the Cauchy
Transform.

Your Mathematica code implements what I\textquotesingle ll call the
\textbf{"Analyst\textquotesingle s Definition"}, which is the one you
linked to earlier for the Kramers-Kronig relations:

\[C_{\text{Ana}}[f](z) = \frac{1}{\pi i} \int \frac{f(t)}{t-z} \, dt\]

Julia\textquotesingle s \texttt{SingularIntegralEquations.cauchy}
function uses the \textbf{"Integral Equation Definition"} common in
singular integral equation theory:

\[C_{\text{IE}}[f](z) = \frac{1}{2\pi i} \int \frac{f(t)}{t-z} \, dt\]

Therefore, the relation is simply:

\[C_{\text{Ana}}[f](z) = 2 \times C_{\text{IE}}[f](z)\]

Your results confirm this: \texttt{0.479351...} (Julia) is half of what
Mathematica gives for the same calculation.

\subsubsection{🧮 Why the Factor of 2 Matters: Connection to the
Plemelj-Sokhotsky
Formula}\label{ux1f9ee-why-the-factor-of-2-matters-connection-to-the-plemelj-sokhotsky-formula}

The "Integral Equation Definition" (\(1/(2\pi i)\)) is often preferred
when connecting to the \textbf{Plemelj-Sokhotsky formula}, which
describes the limiting values of a Cauchy integral as you approach the
real axis. It states:

\[\lim_{\epsilon \to 0^+} \frac{1}{2\pi i} \int \frac{f(t)}{t - (x \pm i\epsilon)} \, dt = \pm \frac{1}{2} f(x) + \frac{1}{2\pi i} \text{P.V.} \int \frac{f(t)}{t - x} \, dt\]

Here, the jump (the \texttt{±\ 1/2\ f(x)}) is nice and clean. The
"Analyst\textquotesingle s Definition" would have a jump of
\texttt{±\ f(x)}.

\subsubsection{🔄 Adapting Your Codes for
Consistency}\label{ux1f504-adapting-your-codes-for-consistency}

You can easily make either code match the other\textquotesingle s
convention.

\textbf{1. To make Mathematica match the Julia result:}\\
Simply multiply your Mathematica integral by \texttt{1/2}.

\begin{Shaded}
\begin{Highlighting}[]
\CommentTok{(* Mathematica code matching Julia\textquotesingle{}s cauchy(f, z) *)}
\NormalTok{CauchyJulia}\OperatorTok{[}\AttributeTok{z\_}\OperatorTok{]} \ExtensionTok{:=} \FunctionTok{Integrate}\OperatorTok{[}\FunctionTok{Exp}\OperatorTok{[}\FunctionTok{t}\OperatorTok{]}\SpecialCharTok{/}\NormalTok{(}\FunctionTok{t} \SpecialCharTok{{-}} \FunctionTok{z}\NormalTok{)}\OperatorTok{,} \OperatorTok{\{}\FunctionTok{t}\OperatorTok{,} \SpecialCharTok{{-}}\DecValTok{1}\OperatorTok{,} \DecValTok{1}\OperatorTok{\}]} \SpecialCharTok{/}\NormalTok{ (}\DecValTok{2} \SpecialCharTok{*} \FunctionTok{Pi} \SpecialCharTok{*} \FunctionTok{I}\NormalTok{)}
\end{Highlighting}
\end{Shaded}

\textbf{2. To make Julia match your Mathematica result:}\\
Multiply Julia\textquotesingle s result by \texttt{2}.

\begin{Shaded}
\begin{Highlighting}[]
\CommentTok{\# Julia code matching your Mathematica definition}
\NormalTok{C\_val\_analyst }\OperatorTok{=} \FloatTok{2} \OperatorTok{*} \FunctionTok{cauchy}\NormalTok{(f, z)}
\CommentTok{\# This should now give: \textasciitilde{}0.958702... + 0.0539166...im}
\end{Highlighting}
\end{Shaded}

\subsubsection{💡 Which One Should You
Use?}\label{ux1f4a1-which-one-should-you-use}

It depends on your field:

\begin{itemize}
\item
  \textbf{Complex Analysis / Kramers-Kronig Relations}: Stick with the
  \textbf{Analyst\textquotesingle s Definition} (\(1/(\pi i)\)). This is
  standard in physics and optics.
\item
  \textbf{Singular Integral Equations / Fracture Mechanics}: The
  \textbf{Integral Equation Definition} (\(1/(2\pi i)\)) is more common.
  This is the world of the Kolosov-Muskhelishvili potentials we
  discussed earlier.
\end{itemize}

The key is to be aware of the definition in the papers or textbooks you
are following.

\subsubsection{✅ Summary and
Verification}\label{ux2705-summary-and-verification}

You can quickly verify the relationship by checking the limiting value
as \texttt{z} approaches the real axis. Using the Plemelj-Sokhotsky
formula with Julia\textquotesingle s definition:

\begin{itemize}
\item
  For \texttt{z\ -\textgreater{}\ x\ +\ i*0+} (from above), the real
  part should approach \texttt{+1/2\ *\ f(x)}.
\item
  For \texttt{z\ -\textgreater{}\ x\ -\ i*0+} (from below), the real
  part should approach \texttt{-1/2\ *\ f(x)}.
\end{itemize}

In summary, the \textbf{difference is not an error but a convention}.
Once you account for the factor of 2, both codes are correct. For your
engineering work in fracture mechanics (which heavily uses singular
integrals), Julia\textquotesingle s built-in convention is likely the
more direct one to use.

If you are comparing results from different sources, always check which
definition they use.

\begin{Shaded}
\begin{Highlighting}[]

\end{Highlighting}
\end{Shaded}

%% Options for packages loaded elsewhere
\PassOptionsToPackage{unicode}{hyperref}
\PassOptionsToPackage{hyphens}{url}
\documentclass[
]{article}
\usepackage{xcolor}
\usepackage{amsmath,amssymb}
\setcounter{secnumdepth}{-\maxdimen} % remove section numbering
\usepackage{iftex}
\ifPDFTeX
  \usepackage[T1]{fontenc}
  \usepackage[utf8]{inputenc}
  \usepackage{textcomp} % provide euro and other symbols
\else % if luatex or xetex
  \usepackage{unicode-math} % this also loads fontspec
  \defaultfontfeatures{Scale=MatchLowercase}
  \defaultfontfeatures[\rmfamily]{Ligatures=TeX,Scale=1}
\fi
\usepackage{lmodern}
\ifPDFTeX\else
  % xetex/luatex font selection
\fi
% Use upquote if available, for straight quotes in verbatim environments
\IfFileExists{upquote.sty}{\usepackage{upquote}}{}
\IfFileExists{microtype.sty}{% use microtype if available
  \usepackage[]{microtype}
  \UseMicrotypeSet[protrusion]{basicmath} % disable protrusion for tt fonts
}{}
\makeatletter
\@ifundefined{KOMAClassName}{% if non-KOMA class
  \IfFileExists{parskip.sty}{%
    \usepackage{parskip}
  }{% else
    \setlength{\parindent}{0pt}
    \setlength{\parskip}{6pt plus 2pt minus 1pt}}
}{% if KOMA class
  \KOMAoptions{parskip=half}}
\makeatother
\usepackage{color}
\usepackage{fancyvrb}
\newcommand{\VerbBar}{|}
\newcommand{\VERB}{\Verb[commandchars=\\\{\}]}
\DefineVerbatimEnvironment{Highlighting}{Verbatim}{commandchars=\\\{\}}
% Add ',fontsize=\small' for more characters per line
\newenvironment{Shaded}{}{}
\newcommand{\AlertTok}[1]{\textcolor[rgb]{1.00,0.00,0.00}{\textbf{#1}}}
\newcommand{\AnnotationTok}[1]{\textcolor[rgb]{0.38,0.63,0.69}{\textbf{\textit{#1}}}}
\newcommand{\AttributeTok}[1]{\textcolor[rgb]{0.49,0.56,0.16}{#1}}
\newcommand{\BaseNTok}[1]{\textcolor[rgb]{0.25,0.63,0.44}{#1}}
\newcommand{\BuiltInTok}[1]{\textcolor[rgb]{0.00,0.50,0.00}{#1}}
\newcommand{\CharTok}[1]{\textcolor[rgb]{0.25,0.44,0.63}{#1}}
\newcommand{\CommentTok}[1]{\textcolor[rgb]{0.38,0.63,0.69}{\textit{#1}}}
\newcommand{\CommentVarTok}[1]{\textcolor[rgb]{0.38,0.63,0.69}{\textbf{\textit{#1}}}}
\newcommand{\ConstantTok}[1]{\textcolor[rgb]{0.53,0.00,0.00}{#1}}
\newcommand{\ControlFlowTok}[1]{\textcolor[rgb]{0.00,0.44,0.13}{\textbf{#1}}}
\newcommand{\DataTypeTok}[1]{\textcolor[rgb]{0.56,0.13,0.00}{#1}}
\newcommand{\DecValTok}[1]{\textcolor[rgb]{0.25,0.63,0.44}{#1}}
\newcommand{\DocumentationTok}[1]{\textcolor[rgb]{0.73,0.13,0.13}{\textit{#1}}}
\newcommand{\ErrorTok}[1]{\textcolor[rgb]{1.00,0.00,0.00}{\textbf{#1}}}
\newcommand{\ExtensionTok}[1]{#1}
\newcommand{\FloatTok}[1]{\textcolor[rgb]{0.25,0.63,0.44}{#1}}
\newcommand{\FunctionTok}[1]{\textcolor[rgb]{0.02,0.16,0.49}{#1}}
\newcommand{\ImportTok}[1]{\textcolor[rgb]{0.00,0.50,0.00}{\textbf{#1}}}
\newcommand{\InformationTok}[1]{\textcolor[rgb]{0.38,0.63,0.69}{\textbf{\textit{#1}}}}
\newcommand{\KeywordTok}[1]{\textcolor[rgb]{0.00,0.44,0.13}{\textbf{#1}}}
\newcommand{\NormalTok}[1]{#1}
\newcommand{\OperatorTok}[1]{\textcolor[rgb]{0.40,0.40,0.40}{#1}}
\newcommand{\OtherTok}[1]{\textcolor[rgb]{0.00,0.44,0.13}{#1}}
\newcommand{\PreprocessorTok}[1]{\textcolor[rgb]{0.74,0.48,0.00}{#1}}
\newcommand{\RegionMarkerTok}[1]{#1}
\newcommand{\SpecialCharTok}[1]{\textcolor[rgb]{0.25,0.44,0.63}{#1}}
\newcommand{\SpecialStringTok}[1]{\textcolor[rgb]{0.73,0.40,0.53}{#1}}
\newcommand{\StringTok}[1]{\textcolor[rgb]{0.25,0.44,0.63}{#1}}
\newcommand{\VariableTok}[1]{\textcolor[rgb]{0.10,0.09,0.49}{#1}}
\newcommand{\VerbatimStringTok}[1]{\textcolor[rgb]{0.25,0.44,0.63}{#1}}
\newcommand{\WarningTok}[1]{\textcolor[rgb]{0.38,0.63,0.69}{\textbf{\textit{#1}}}}
\setlength{\emergencystretch}{3em} % prevent overfull lines
\providecommand{\tightlist}{%
  \setlength{\itemsep}{0pt}\setlength{\parskip}{0pt}}
\usepackage{bookmark}
\IfFileExists{xurl.sty}{\usepackage{xurl}}{} % add URL line breaks if available
\urlstyle{same}
\hypersetup{
  hidelinks,
  pdfcreator={LaTeX via pandoc}}

\author{}
\date{}

\begin{document}

The difference of exactly a factor of 1/2 arises because
Julia\textquotesingle s \texttt{SingularIntegralEquations.jl} and the
standard Mathematica \texttt{Integrate} function are using
\textbf{different, but equally valid, definitions} of the Cauchy
Transform.

Your Mathematica code implements what I\textquotesingle ll call the
\textbf{"Analyst\textquotesingle s Definition"}, which is the one you
linked to earlier for the Kramers-Kronig relations:

\[C_{\text{Ana}}[f](z) = \frac{1}{\pi i} \int \frac{f(t)}{t-z} \, dt\]

Julia\textquotesingle s \texttt{SingularIntegralEquations.cauchy}
function uses the \textbf{"Integral Equation Definition"} common in
singular integral equation theory:

\[C_{\text{IE}}[f](z) = \frac{1}{2\pi i} \int \frac{f(t)}{t-z} \, dt\]

Therefore, the relation is simply:

\[C_{\text{Ana}}[f](z) = 2 \times C_{\text{IE}}[f](z)\]

Your results confirm this: \texttt{0.479351...} (Julia) is half of what
Mathematica gives for the same calculation.

\subsubsection{🧮 Why the Factor of 2 Matters: Connection to the
Plemelj-Sokhotsky
Formula}\label{ux1f9ee-why-the-factor-of-2-matters-connection-to-the-plemelj-sokhotsky-formula}

The "Integral Equation Definition" (\(1/(2\pi i)\)) is often preferred
when connecting to the \textbf{Plemelj-Sokhotsky formula}, which
describes the limiting values of a Cauchy integral as you approach the
real axis. It states:

\[\lim_{\epsilon \to 0^+} \frac{1}{2\pi i} \int \frac{f(t)}{t - (x \pm i\epsilon)} \, dt = \pm \frac{1}{2} f(x) + \frac{1}{2\pi i} \text{P.V.} \int \frac{f(t)}{t - x} \, dt\]

Here, the jump (the \texttt{±\ 1/2\ f(x)}) is nice and clean. The
"Analyst\textquotesingle s Definition" would have a jump of
\texttt{±\ f(x)}.

\subsubsection{🔄 Adapting Your Codes for
Consistency}\label{ux1f504-adapting-your-codes-for-consistency}

You can easily make either code match the other\textquotesingle s
convention.

\textbf{1. To make Mathematica match the Julia result:}\\
Simply multiply your Mathematica integral by \texttt{1/2}.

\begin{Shaded}
\begin{Highlighting}[]
\CommentTok{(* Mathematica code matching Julia\textquotesingle{}s cauchy(f, z) *)}
\NormalTok{CauchyJulia}\OperatorTok{[}\AttributeTok{z\_}\OperatorTok{]} \ExtensionTok{:=} \FunctionTok{Integrate}\OperatorTok{[}\FunctionTok{Exp}\OperatorTok{[}\FunctionTok{t}\OperatorTok{]}\SpecialCharTok{/}\NormalTok{(}\FunctionTok{t} \SpecialCharTok{{-}} \FunctionTok{z}\NormalTok{)}\OperatorTok{,} \OperatorTok{\{}\FunctionTok{t}\OperatorTok{,} \SpecialCharTok{{-}}\DecValTok{1}\OperatorTok{,} \DecValTok{1}\OperatorTok{\}]} \SpecialCharTok{/}\NormalTok{ (}\DecValTok{2} \SpecialCharTok{*} \FunctionTok{Pi} \SpecialCharTok{*} \FunctionTok{I}\NormalTok{)}
\end{Highlighting}
\end{Shaded}

\textbf{2. To make Julia match your Mathematica result:}\\
Multiply Julia\textquotesingle s result by \texttt{2}.

\begin{Shaded}
\begin{Highlighting}[]
\CommentTok{\# Julia code matching your Mathematica definition}
\NormalTok{C\_val\_analyst }\OperatorTok{=} \FloatTok{2} \OperatorTok{*} \FunctionTok{cauchy}\NormalTok{(f, z)}
\CommentTok{\# This should now give: \textasciitilde{}0.958702... + 0.0539166...im}
\end{Highlighting}
\end{Shaded}

\subsubsection{💡 Which One Should You
Use?}\label{ux1f4a1-which-one-should-you-use}

It depends on your field:

\begin{itemize}
\item
  \textbf{Complex Analysis / Kramers-Kronig Relations}: Stick with the
  \textbf{Analyst\textquotesingle s Definition} (\(1/(\pi i)\)). This is
  standard in physics and optics.
\item
  \textbf{Singular Integral Equations / Fracture Mechanics}: The
  \textbf{Integral Equation Definition} (\(1/(2\pi i)\)) is more common.
  This is the world of the Kolosov-Muskhelishvili potentials we
  discussed earlier.
\end{itemize}

The key is to be aware of the definition in the papers or textbooks you
are following.

\subsubsection{✅ Summary and
Verification}\label{ux2705-summary-and-verification}

You can quickly verify the relationship by checking the limiting value
as \texttt{z} approaches the real axis. Using the Plemelj-Sokhotsky
formula with Julia\textquotesingle s definition:

\begin{itemize}
\item
  For \texttt{z\ -\textgreater{}\ x\ +\ i*0+} (from above), the real
  part should approach \texttt{+1/2\ *\ f(x)}.
\item
  For \texttt{z\ -\textgreater{}\ x\ -\ i*0+} (from below), the real
  part should approach \texttt{-1/2\ *\ f(x)}.
\end{itemize}

In summary, the \textbf{difference is not an error but a convention}.
Once you account for the factor of 2, both codes are correct. For your
engineering work in fracture mechanics (which heavily uses singular
integrals), Julia\textquotesingle s built-in convention is likely the
more direct one to use.

If you are comparing results from different sources, always check which
definition they use.

\begin{Shaded}
\begin{Highlighting}[]

\end{Highlighting}
\end{Shaded}

%% Options for packages loaded elsewhere
\PassOptionsToPackage{unicode}{hyperref}
\PassOptionsToPackage{hyphens}{url}
\documentclass[
]{article}
\usepackage{xcolor}
\usepackage{amsmath,amssymb}
\setcounter{secnumdepth}{-\maxdimen} % remove section numbering
\usepackage{iftex}
\ifPDFTeX
  \usepackage[T1]{fontenc}
  \usepackage[utf8]{inputenc}
  \usepackage{textcomp} % provide euro and other symbols
\else % if luatex or xetex
  \usepackage{unicode-math} % this also loads fontspec
  \defaultfontfeatures{Scale=MatchLowercase}
  \defaultfontfeatures[\rmfamily]{Ligatures=TeX,Scale=1}
\fi
\usepackage{lmodern}
\ifPDFTeX\else
  % xetex/luatex font selection
\fi
% Use upquote if available, for straight quotes in verbatim environments
\IfFileExists{upquote.sty}{\usepackage{upquote}}{}
\IfFileExists{microtype.sty}{% use microtype if available
  \usepackage[]{microtype}
  \UseMicrotypeSet[protrusion]{basicmath} % disable protrusion for tt fonts
}{}
\makeatletter
\@ifundefined{KOMAClassName}{% if non-KOMA class
  \IfFileExists{parskip.sty}{%
    \usepackage{parskip}
  }{% else
    \setlength{\parindent}{0pt}
    \setlength{\parskip}{6pt plus 2pt minus 1pt}}
}{% if KOMA class
  \KOMAoptions{parskip=half}}
\makeatother
\usepackage{color}
\usepackage{fancyvrb}
\newcommand{\VerbBar}{|}
\newcommand{\VERB}{\Verb[commandchars=\\\{\}]}
\DefineVerbatimEnvironment{Highlighting}{Verbatim}{commandchars=\\\{\}}
% Add ',fontsize=\small' for more characters per line
\newenvironment{Shaded}{}{}
\newcommand{\AlertTok}[1]{\textcolor[rgb]{1.00,0.00,0.00}{\textbf{#1}}}
\newcommand{\AnnotationTok}[1]{\textcolor[rgb]{0.38,0.63,0.69}{\textbf{\textit{#1}}}}
\newcommand{\AttributeTok}[1]{\textcolor[rgb]{0.49,0.56,0.16}{#1}}
\newcommand{\BaseNTok}[1]{\textcolor[rgb]{0.25,0.63,0.44}{#1}}
\newcommand{\BuiltInTok}[1]{\textcolor[rgb]{0.00,0.50,0.00}{#1}}
\newcommand{\CharTok}[1]{\textcolor[rgb]{0.25,0.44,0.63}{#1}}
\newcommand{\CommentTok}[1]{\textcolor[rgb]{0.38,0.63,0.69}{\textit{#1}}}
\newcommand{\CommentVarTok}[1]{\textcolor[rgb]{0.38,0.63,0.69}{\textbf{\textit{#1}}}}
\newcommand{\ConstantTok}[1]{\textcolor[rgb]{0.53,0.00,0.00}{#1}}
\newcommand{\ControlFlowTok}[1]{\textcolor[rgb]{0.00,0.44,0.13}{\textbf{#1}}}
\newcommand{\DataTypeTok}[1]{\textcolor[rgb]{0.56,0.13,0.00}{#1}}
\newcommand{\DecValTok}[1]{\textcolor[rgb]{0.25,0.63,0.44}{#1}}
\newcommand{\DocumentationTok}[1]{\textcolor[rgb]{0.73,0.13,0.13}{\textit{#1}}}
\newcommand{\ErrorTok}[1]{\textcolor[rgb]{1.00,0.00,0.00}{\textbf{#1}}}
\newcommand{\ExtensionTok}[1]{#1}
\newcommand{\FloatTok}[1]{\textcolor[rgb]{0.25,0.63,0.44}{#1}}
\newcommand{\FunctionTok}[1]{\textcolor[rgb]{0.02,0.16,0.49}{#1}}
\newcommand{\ImportTok}[1]{\textcolor[rgb]{0.00,0.50,0.00}{\textbf{#1}}}
\newcommand{\InformationTok}[1]{\textcolor[rgb]{0.38,0.63,0.69}{\textbf{\textit{#1}}}}
\newcommand{\KeywordTok}[1]{\textcolor[rgb]{0.00,0.44,0.13}{\textbf{#1}}}
\newcommand{\NormalTok}[1]{#1}
\newcommand{\OperatorTok}[1]{\textcolor[rgb]{0.40,0.40,0.40}{#1}}
\newcommand{\OtherTok}[1]{\textcolor[rgb]{0.00,0.44,0.13}{#1}}
\newcommand{\PreprocessorTok}[1]{\textcolor[rgb]{0.74,0.48,0.00}{#1}}
\newcommand{\RegionMarkerTok}[1]{#1}
\newcommand{\SpecialCharTok}[1]{\textcolor[rgb]{0.25,0.44,0.63}{#1}}
\newcommand{\SpecialStringTok}[1]{\textcolor[rgb]{0.73,0.40,0.53}{#1}}
\newcommand{\StringTok}[1]{\textcolor[rgb]{0.25,0.44,0.63}{#1}}
\newcommand{\VariableTok}[1]{\textcolor[rgb]{0.10,0.09,0.49}{#1}}
\newcommand{\VerbatimStringTok}[1]{\textcolor[rgb]{0.25,0.44,0.63}{#1}}
\newcommand{\WarningTok}[1]{\textcolor[rgb]{0.38,0.63,0.69}{\textbf{\textit{#1}}}}
\setlength{\emergencystretch}{3em} % prevent overfull lines
\providecommand{\tightlist}{%
  \setlength{\itemsep}{0pt}\setlength{\parskip}{0pt}}
\usepackage{bookmark}
\IfFileExists{xurl.sty}{\usepackage{xurl}}{} % add URL line breaks if available
\urlstyle{same}
\hypersetup{
  hidelinks,
  pdfcreator={LaTeX via pandoc}}

\author{}
\date{}

\begin{document}

The difference of exactly a factor of 1/2 arises because
Julia\textquotesingle s \texttt{SingularIntegralEquations.jl} and the
standard Mathematica \texttt{Integrate} function are using
\textbf{different, but equally valid, definitions} of the Cauchy
Transform.

Your Mathematica code implements what I\textquotesingle ll call the
\textbf{"Analyst\textquotesingle s Definition"}, which is the one you
linked to earlier for the Kramers-Kronig relations:

\[C_{\text{Ana}}[f](z) = \frac{1}{\pi i} \int \frac{f(t)}{t-z} \, dt\]

Julia\textquotesingle s \texttt{SingularIntegralEquations.cauchy}
function uses the \textbf{"Integral Equation Definition"} common in
singular integral equation theory:

\[C_{\text{IE}}[f](z) = \frac{1}{2\pi i} \int \frac{f(t)}{t-z} \, dt\]

Therefore, the relation is simply:

\[C_{\text{Ana}}[f](z) = 2 \times C_{\text{IE}}[f](z)\]

Your results confirm this: \texttt{0.479351...} (Julia) is half of what
Mathematica gives for the same calculation.

\subsubsection{🧮 Why the Factor of 2 Matters: Connection to the
Plemelj-Sokhotsky
Formula}\label{ux1f9ee-why-the-factor-of-2-matters-connection-to-the-plemelj-sokhotsky-formula}

The "Integral Equation Definition" (\(1/(2\pi i)\)) is often preferred
when connecting to the \textbf{Plemelj-Sokhotsky formula}, which
describes the limiting values of a Cauchy integral as you approach the
real axis. It states:

\[\lim_{\epsilon \to 0^+} \frac{1}{2\pi i} \int \frac{f(t)}{t - (x \pm i\epsilon)} \, dt = \pm \frac{1}{2} f(x) + \frac{1}{2\pi i} \text{P.V.} \int \frac{f(t)}{t - x} \, dt\]

Here, the jump (the \texttt{±\ 1/2\ f(x)}) is nice and clean. The
"Analyst\textquotesingle s Definition" would have a jump of
\texttt{±\ f(x)}.

\subsubsection{🔄 Adapting Your Codes for
Consistency}\label{ux1f504-adapting-your-codes-for-consistency}

You can easily make either code match the other\textquotesingle s
convention.

\textbf{1. To make Mathematica match the Julia result:}\\
Simply multiply your Mathematica integral by \texttt{1/2}.

\begin{Shaded}
\begin{Highlighting}[]
\CommentTok{(* Mathematica code matching Julia\textquotesingle{}s cauchy(f, z) *)}
\NormalTok{CauchyJulia}\OperatorTok{[}\AttributeTok{z\_}\OperatorTok{]} \ExtensionTok{:=} \FunctionTok{Integrate}\OperatorTok{[}\FunctionTok{Exp}\OperatorTok{[}\FunctionTok{t}\OperatorTok{]}\SpecialCharTok{/}\NormalTok{(}\FunctionTok{t} \SpecialCharTok{{-}} \FunctionTok{z}\NormalTok{)}\OperatorTok{,} \OperatorTok{\{}\FunctionTok{t}\OperatorTok{,} \SpecialCharTok{{-}}\DecValTok{1}\OperatorTok{,} \DecValTok{1}\OperatorTok{\}]} \SpecialCharTok{/}\NormalTok{ (}\DecValTok{2} \SpecialCharTok{*} \FunctionTok{Pi} \SpecialCharTok{*} \FunctionTok{I}\NormalTok{)}
\end{Highlighting}
\end{Shaded}

\textbf{2. To make Julia match your Mathematica result:}\\
Multiply Julia\textquotesingle s result by \texttt{2}.

\begin{Shaded}
\begin{Highlighting}[]
\CommentTok{\# Julia code matching your Mathematica definition}
\NormalTok{C\_val\_analyst }\OperatorTok{=} \FloatTok{2} \OperatorTok{*} \FunctionTok{cauchy}\NormalTok{(f, z)}
\CommentTok{\# This should now give: \textasciitilde{}0.958702... + 0.0539166...im}
\end{Highlighting}
\end{Shaded}

\subsubsection{💡 Which One Should You
Use?}\label{ux1f4a1-which-one-should-you-use}

It depends on your field:

\begin{itemize}
\item
  \textbf{Complex Analysis / Kramers-Kronig Relations}: Stick with the
  \textbf{Analyst\textquotesingle s Definition} (\(1/(\pi i)\)). This is
  standard in physics and optics.
\item
  \textbf{Singular Integral Equations / Fracture Mechanics}: The
  \textbf{Integral Equation Definition} (\(1/(2\pi i)\)) is more common.
  This is the world of the Kolosov-Muskhelishvili potentials we
  discussed earlier.
\end{itemize}

The key is to be aware of the definition in the papers or textbooks you
are following.

\subsubsection{✅ Summary and
Verification}\label{ux2705-summary-and-verification}

You can quickly verify the relationship by checking the limiting value
as \texttt{z} approaches the real axis. Using the Plemelj-Sokhotsky
formula with Julia\textquotesingle s definition:

\begin{itemize}
\item
  For \texttt{z\ -\textgreater{}\ x\ +\ i*0+} (from above), the real
  part should approach \texttt{+1/2\ *\ f(x)}.
\item
  For \texttt{z\ -\textgreater{}\ x\ -\ i*0+} (from below), the real
  part should approach \texttt{-1/2\ *\ f(x)}.
\end{itemize}

In summary, the \textbf{difference is not an error but a convention}.
Once you account for the factor of 2, both codes are correct. For your
engineering work in fracture mechanics (which heavily uses singular
integrals), Julia\textquotesingle s built-in convention is likely the
more direct one to use.

If you are comparing results from different sources, always check which
definition they use.

\begin{Shaded}
\begin{Highlighting}[]

\end{Highlighting}
\end{Shaded}

%\include{Cauchy}
\end{document}

\end{document}

\end{document}

\end{document}
